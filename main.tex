%!TEX encoding = UTF-8 Unicode
%!TEX program = xelatex
% -*- coding: utf-8 -*-
%%
%%
%%
%%
%%
%%
%%  本模板可以使用以下两种方式编译:
%%
%%     1. PDFLaTeX
%%
%%     2. XeLaTeX [推荐]
%%
%%  注意:
%%    1. 在改变编译方式前应先删除 *.toc 和 *.aux 文件,
%%       因为不同编译方式产生的辅助文件格式可能并不相同。
%%
%%
\documentclass[12pt,openright]{book}


\usepackage{ifxetex}
\ifxetex
  \usepackage[bookmarksnumbered]{hyperref}
\else
  \usepackage[unicode,bookmarksnumbered]{hyperref}
\fi

\usepackage[emptydoublepage]{NKThesis}   % 中文
%\usepackage[emptydoublepage,English]{NKThesis} % 英文

%   根据需要选择 biblatex 宏包选项.
\usepackage[backend = bibtex8, defernumbers = true,  sorting=none,  style = nkthesis]{biblatex}
\hypersetup{colorlinks=true,
            pdfborder=0 0 1,
            citecolor=black,
            linkcolor=black}
\usepackage{tikz}

\addbibresource{nkthesis.bib}
\DeclareBibliographyCategory{cited}
\AtEveryCitekey{\addtocategory{cited}{\thefield{entrykey}}}

\graphicspath{{image/}}

\includeonly{
abstract,
manual,
tikz,
acknowledgements,
references,
appendices,
resume
}
\newtheorem{Theorem}{\hskip 2em 定理}[chapter]
\newtheorem{Lemma}[Theorem]{\hskip 2em 引理}
\newtheorem{Corollary}[Theorem]{\hskip 2em 推论}
\newtheorem{Proposition}[Theorem]{\hskip 2em 命题}
\newtheorem{Definition}[Theorem]{\hskip 2em 定义}
\newtheorem{Example}[Theorem]{\hskip 2em 例}
\begin{document}

%  设置基本信息
%  注意:  逗号`,'是项目分隔符. 如果某一项的值出现逗号, 应放在花括号内, 如 {,}
%
\NKTsetup{%
	论文题目(中文) = 学位论文的标题,
	副标题        = ——论文的副标题,
	论文题目(英文) = Title of thia Thesis in English,
	论文作者       = ,
	学号           = ,
	指导教师       = ,
	申请学位       = ,
	培养单位       = ,
	学科专业       = ,
	研究方向       = ,
	中图分类号     = ,
	UDC            = ,
	学校代码       = 10055,
	密级           = 公开,
	% 公开 | 限制 | 秘密 | 机密, 若为公开, 不填以下三项
	保密期限       = ,
	审批表编号     = ,
	批准日期       = ,
	论文完成时间   = 二〇一八年三月,
	答辩日期       = ,
	论文类别       = 学历硕士,
	% 博士 | 学历硕士 | 硕士专业学位 | 高校教师 | 同等学力硕士
	院/系/所       = ,
	专业           = ,
	联系电话       = ,
	Email          = ,
	通讯地址(邮编) = ,
	备注           = }


% -*- coding: utf-8 -*-


\begin{zhaiyao}

这里输入中文摘要。
\newpage

中文摘要ABF
\end{zhaiyao}




\begin{guanjianci}
毕业论文;模板
\end{guanjianci}



\begin{abstract}


This is the abstract.

\end{abstract}



\begin{keywords}
Thesis; template
\end{keywords} 
\tableofcontents
\input{./tex/manual}
% -*- coding: utf-8 -*-


\chapter{The Tikz Package}


The {\scshape pdf}\ package, where ``{\scshape pdf}'' is supposed to mean ``portable
graphics format'' (or ``pretty, good, functional'' if you
prefer\dots), is a package for creating graphics in an ``inline''
manner. It defines a number of \TeX\ commands that draw
graphics. For example, the code \verb|\tikz \draw (0pt,0pt) -- (20pt,6pt);|
yields the line \tikz \draw (0pt,0pt) -- (20pt,6pt); and the code \verb|\tikz \fill[orange] (1ex,1ex) circle (1ex);| yields \tikz
\fill[orange] (1ex,1ex) circle (1ex);.

In a sense, when you use {\scshape pdf}\ you ``program'' your graphics, just
as you ``program'' your document when you use \TeX.  You get all
the advantages of the ``\TeX-approach to typesetting'' for your
graphics: quick creation of simple graphics, precise positioning, the
use of macros, often superior typography. You also inherit all the
disadvantages: steep learning curve, no \textsc{wysiwyg}, small
changes require a long recompilation time, and the code does not
really ``show'' how things will look like.




% -*- coding: utf-8 -*-

\def\bibrangedash{ $\sim$ }
\printbibliography [ category = cited]


% -*- coding: utf-8 -*-

%\makeschapterhead{致谢}
\chapter*{致谢}
感谢您使用本模板。

\input{./tex/appendices}
% -*- coding: utf-8 -*-


\chapter*{个人简历}

\end{document}
